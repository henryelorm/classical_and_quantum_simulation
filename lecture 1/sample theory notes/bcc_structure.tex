% Setting up the document class for a standard article
\documentclass[a4paper,12pt]{article}

% Including essential LaTeX packages for mathematical typesetting
\usepackage{amsmath} % Advanced math typesetting
\usepackage{amsfonts} % Math fonts
\usepackage{amssymb} % Additional math symbols
\usepackage{geometry} % Page layout customization
\usepackage{tensor} % Tensor notation
\usepackage{physics} % Physics-related macros
\usepackage{enumitem} % Customized lists
\usepackage{graphicx} % Graphics inclusion (not used here but included for flexibility)

% Configuring page geometry for balanced margins
\geometry{margin=1in}

% Setting up fonts compatible with PDFLaTeX
\usepackage[T1]{fontenc}
\usepackage{lmodern} % Latin Modern font, scalable and reliable
\usepackage{mathptmx} % Times-like math font for consistency

% Defining custom commands for clarity
\newcommand{\bvec}[1]{\mathbf{#1}} % Bold vector notation
\newcommand{\eps}{\epsilon} % Levi-Civita symbol shorthand
\newcommand{\delt}{\delta} % Kronecker delta shorthand

% Starting the document
\begin{document}
	
	% Title and section for the proof
	\title{bcc Lattice and Primitive Unit Cell}
	\maketitle
	
	% Section: Tensor Representation of BCC Lattice
	\section{bcc Lattice (beware of tensors)}
	
	% Describing the conventional unit cell
	The body-centered cubic (BCC) lattice is defined by a conventional cubic unit cell with lattice constant \( a \). The basis vectors of the conventional unit cell in Cartesian coordinates are:
	
	\begin{equation}
		\bvec{a}_1 = a (1, 0, 0), \quad \bvec{a}_2 = a (0, 1, 0), \quad \bvec{a}_3 = a (0, 0, 1)
	\end{equation}
	
	In tensor notation, these vectors are contravariant, expressed as \(\bvec{a}_i = a_i^\mu \bvec{e}_\mu\), where \(\bvec{e}_\mu\) (\(\mu = 1, 2, 3\)) are the Cartesian basis vectors, and:
	
	\begin{equation}
		a_i^\mu = a \delt_i^\mu
	\end{equation}
	
	where \(\delt_i^\mu\) is the Kronecker delta (\(\delt_i^\mu = 1\) if \(i = \mu\), else 0).
	
	% Basis of the BCC crystal structure
	The BCC crystal structure has a basis of two atoms per conventional unit cell, located at:
	
	\begin{equation}
		\bvec{r}_1 = (0, 0, 0), \quad \bvec{r}_2 = \frac{a}{2} (1, 1, 1)
	\end{equation}
	
	In tensor form:
	
	\begin{equation}
		r_1^\mu = (0, 0, 0), \quad r_2^\mu = \frac{a}{2} (1, 1, 1)
	\end{equation}
	
	The position of any atom in the crystal is:
	
	\begin{equation}
		\bvec{R} = n^i \bvec{a}_i + \bvec{r}_j, \quad R^\mu = n^i a_i^\mu + r_j^\mu
	\end{equation}
	
	where \(n^i \in \mathbb{Z}\) and \(j = 1, 2\). Substituting:
	
	\begin{equation}
		R^\mu = n^i a \delt_i^\mu + r_j^\mu = a (n^1 \delt_1^\mu + n^2 \delt_2^\mu + n^3 \delt_3^\mu) + r_j^\mu
	\end{equation}
	
	For \(j = 1\):
	
	\begin{equation}
		R^\mu = a (n^1, n^2, n^3)
	\end{equation}
	
	For \(j = 2\):
	
	\begin{equation}
		R^\mu = a \left( n^1 + \frac{1}{2}, n^2 + \frac{1}{2}, n^3 + \frac{1}{2} \right)
	\end{equation}
	
	This generates all BCC lattice positions.
	
	% Section: Metric Tensor and Volume of Conventional Unit Cell
	\section{Metric Tensor and Volume of Conventional Unit Cell}
	
	% Computing the metric tensor
	The metric tensor is:
	
	\begin{equation}
		g_{ij} = \bvec{a}_i \cdot \bvec{a}_j = a_i^\mu a_j^\nu \delt_{\mu\nu}
	\end{equation}
	
	Since \(\bvec{a}_i\) are orthogonal:
	
	\begin{equation}
		g_{ij} = a^2 \delt_{ij}
	\end{equation}
	
	The volume is given by the scalar triple product:
	
	\begin{equation}
		V = |\bvec{a}_1 \cdot (\bvec{a}_2 \times \bvec{a}_3)| = |\eps^{\mu\nu\lambda} a_{1\mu} a_{2\nu} a_{3\lambda}|
	\end{equation}
	
	Substituting:
	
	\begin{equation}
		V = \eps^{\mu\nu\lambda} (a \delt_{1\mu}) (a \delt_{2\nu}) (a \delt_{3\lambda}) = a^3 \eps^{123} = a^3
	\end{equation}
	
	The volume is \(V = a^3\).
	
	% Section: Primitive Unit Cell
	\section{Primitive Unit Cell}
	
	% Defining primitive basis vectors
	The primitive unit cell contains one lattice point. Suitable primitive vectors are:
	
	\begin{equation}
		\bvec{b}_1 = \frac{a}{2} (1, 1, -1), \quad \bvec{b}_2 = \frac{a}{2} (-1, 1, 1), \quad \bvec{b}_3 = \frac{a}{2} (1, -1, 1)
	\end{equation}
	
	In tensor notation:
	
	\begin{equation}
		b_i^\mu = \frac{a}{2} \begin{cases}
			(1, 1, -1) & i = 1 \\
			(-1, 1, 1) & i = 2 \\
			(1, -1, 1) & i = 3
		\end{cases}
	\end{equation}
	
	Lattice points are generated by:
	
	\begin{equation}
		R^\mu = m^i b_i^\mu, \quad m^i \in \mathbb{Z}
	\end{equation}
	
	% Section: Metric Tensor for Primitive Unit Cell
	\section{Metric Tensor for Primitive Unit Cell}
	
	The metric tensor is:
	
	\begin{equation}
		g_{ij} = \bvec{b}_i \cdot \bvec{b}_j = b_i^\mu b_j^\nu \delt_{\mu\nu}
	\end{equation}
	
	Computing:
	
	\begin{align}
		g_{11} &= \frac{a^2}{4} (1^2 + 1^2 + (-1)^2) = \frac{3a^2}{4} \\
		g_{12} &= \frac{a^2}{4} [(1)(-1) + (1)(1) + (-1)(1)] = -\frac{a^2}{4} \\
		g_{13} &= \frac{a^2}{4} [(1)(1) + (1)(-1) + (-1)(1)] = -\frac{a^2}{4} \\
		g_{22} &= \frac{a^2}{4} [(-1)^2 + 1^2 + 1^2] = \frac{3a^2}{4} \\
		g_{23} &= \frac{a^2}{4} [(-1)(1) + (1)(-1) + (1)(1)] = -\frac{a^2}{4} \\
		g_{33} &= \frac{a^2}{4} [1^2 + (-1)^2 + 1^2] = \frac{3a^2}{4}
	\end{align}
	
	Thus:
	
	\begin{equation}
		g_{ij} = \frac{a^2}{4} \begin{pmatrix}
			3 & -1 & -1 \\
			-1 & 3 & -1 \\
			-1 & -1 & 3
		\end{pmatrix}
	\end{equation}
	
	% Section: Volume of Primitive Unit Cell
	\section{Volume of Primitive Unit Cell}
	
	The volume is:
	
	\begin{equation}
		V = |\bvec{b}_1 \cdot (\bvec{b}_2 \times \bvec{b}_3)| = |\eps^{\mu\nu\lambda} b_{1\mu} b_{2\nu} b_{3\lambda}|
	\end{equation}
	
	Compute:
	
	\begin{align}
		\bvec{b}_2 \times \bvec{b}_3 &= \frac{a^2}{4} \begin{vmatrix}
			\hat{\bvec{e}}_1 & \hat{\bvec{e}}_2 & \hat{\bvec{e}}_3 \\
			-1 & 1 & 1 \\
			1 & -1 & 1
		\end{vmatrix} = \frac{a^2}{4} [2 \hat{\bvec{e}}_1 + 2 \hat{\bvec{e}}_2 + 0 \hat{\bvec{e}}_3] = \frac{a^2}{2} (1, 1, 0)
	\end{align}
	
	\begin{align}
		V &= \left| \frac{a}{2} (1, 1, -1) \cdot \frac{a^2}{2} (1, 1, 0) \right| = \frac{a^3}{4} |1 \cdot 1 + 1 \cdot 1 + (-1) \cdot 0| = \frac{a^3}{4} \cdot 2 = \frac{a^3}{2}
	\end{align}
	
	The volume is \(\frac{a^3}{2}\), half the conventional cell volume, confirming one lattice point.
	
	% Section: Basis in Primitive Unit Cell
	\section{Basis in Primitive Unit Cell}
	
	The primitive unit cell has one lattice point with a basis of one atom at:
	
	\begin{equation}
		r^\mu = (0, 0, 0)
	\end{equation}
	
	Positions are:
	
	\begin{equation}
		R^\mu = m^i b_i^\mu
	\end{equation}
	
	% Section: Transformation Between Cells
	\section{Transformation Between Conventional and Primitive Cells}
	
	Express primitive vectors as:
	
	\begin{align}
		\bvec{b}_1 &= \frac{1}{2} (\bvec{a}_1 + \bvec{a}_2 - \bvec{a}_3) \\
		\bvec{b}_2 &= \frac{1}{2} (-\bvec{a}_1 + \bvec{a}_2 + \bvec{a}_3) \\
		\bvec{b}_3 &= \frac{1}{2} (\bvec{a}_1 - \bvec{a}_2 + \bvec{a}_3)
	\end{align}
	
	In tensor form:
	
	\begin{equation}
		b_i^\mu = T_i^j a_j^\mu, \quad T_i^j = \frac{1}{2} \begin{pmatrix}
			1 & 1 & -1 \\
			-1 & 1 & 1 \\
			1 & -1 & 1
		\end{pmatrix}
	\end{equation}
	
	The volume ratio is:
	
	\begin{equation}
		\det(T) = \frac{1}{8} \begin{vmatrix}
			1 & 1 & -1 \\
			-1 & 1 & 1 \\
			1 & -1 & 1
		\end{vmatrix} = \frac{1}{8} [1(1 \cdot 1 - 1 \cdot (-1)) - 1((-1) \cdot 1 - 1 \cdot 1) + (-1)((-1) \cdot (-1) - 1 \cdot 1)] = \frac{1}{2}
	\end{equation}
	
	Thus:
	
	\begin{equation}
		V_{\text{primitive}} = |\det(T)| V_{\text{conventional}} = \frac{1}{2} \cdot a^3 = \frac{a^3}{2}
	\end{equation}
	
	% Section: Conclusion
	\section{Conclusion}
	
	The BCC lattice is described by:
	
	\begin{itemize}
		\item \textbf{Conventional unit cell}: Basis vectors \(a_i^\mu = a \delt_i^\mu\), two-atom basis at \(r_1^\mu = (0, 0, 0)\), \(r_2^\mu = \frac{a}{2} (1, 1, 1)\). Volume \(a^3\), metric tensor \(g_{ij} = a^2 \delt_{ij}\).
		\item \textbf{Primitive unit cell}: Basis vectors \(b_1^\mu = \frac{a}{2} (1, 1, -1)\), \(b_2^\mu = \frac{a}{2} (-1, 1, 1)\), \(b_3^\mu = \frac{a}{2} (1, -1, 1)\), single-atom basis at \(r^\mu = (0, 0, 0)\). Volume \(\frac{a^3}{2}\), metric tensor \(g_{ij} = \frac{a^2}{4} \begin{pmatrix} 3 & -1 & -1 \\ -1 & 3 & -1 \\ -1 & -1 & 3 \end{pmatrix}\).
	\end{itemize}
	
	The transformation \(T_i^j\) and volume calculations prove the primitive cell contains one lattice point, consistent with the BCC structure.
	
\end{document}